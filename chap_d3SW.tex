\chapter{Lower bounds for depth-3 circuits}\label{chap:d3SW}

In this chapter, we shall see the lower bound of Shpilka and Wigderson~\cite{sw2001} for non-homogeneous depth-$3$ circuits over arbitrary fields.
The main theorem of this section would be the following \emph{quadratic} lower bound. 

\begin{theorem}[\cite{sw2001}]\label{thm:SW-SPS-main-thm}
Any $\SPS$ circuit that computes the polynomial $\ESym_d$, for $d = n/100$ must have at least $\Omega(n^2)$ wires. 
\end{theorem}

In fact, $\ESym_d$ can indeed be computed by a $O(n^2)$-sized  $\SPS$ circuit over a characteristic zero field (\autoref{ex:ben-or-trick}) and hence the above result is tight for $\ESym_d$. 
Until recently, this was the best lower bound we knew for the class of general $\Sigma\Pi\Sigma$ circuits but a very recent result of Kayal, Saha and Tavenas~\cite{kst16} has improved this to an almost cubic lower bound (for a different explicit family of polynomials) which we shall see at a later point.
This chapter however shall focus on the proof of the above theorem.

\section{Lower bounds for hom. $\SPS$ circuits \cite{nw1997}}

Let us first consider the following restricted question.
Say we want to compute an $n$-variate degree $d$ polynomial $f$ using a $\SPS$ circuit, but under the restriction that all intermediate computations have degree at most $d$.
The class of such circuits are denoted by $\Sigma\Pi^{[d]}\Sigma$ circuits.
Can we prove lower bounds for this class first?

Indeed, and in fact we have already seen how to in \autoref{chap:simpleLBs}. But it would be good to recall the method again as we would be using this heavily. 

\begin{theorem}[\cite{nw1997}]
Any $\Sigma\Pi^{[d]}\Sigma$ circuit that computes $\ESym_d$ must have size $\frac{\binom{n}{d/2}}{2^d}$. 
\end{theorem}
Thus, if $n = 3d$, this gives an exponential lower bound. 

\begin{proof}
For a polynomial $f$, define the \emph{dimension of $k$-th order partial derivatives}, denoted by $\CM{NW}_k(f)$, as follows
\[
\CM{NW}_k(f) \spaced{=} \dim \partial^{=k}(f)
\]
\begin{claim}\label{claim:d3-nw-upperbound}
  If $f = \ell_1 \cdots \ell_d$ where each $\ell_i$ is a linear polynomial, then $\CM{NW}_k(f) \leq \binom{d}{k}$. 
  Thus, if $f = \ell_{11}\cdots \ell_{1d} + \cdots + \ell_{s1}\cdots \ell_{sd}$, then $\CM{NW}_k(f) \leq s \cdot \binom{d}{k}$. 
\end{claim}

\noindent
A fact that we would need here but won't prove is that $\ESym_d$ has large space of partial derivatives. 

\begin{claim}
\[
\CM{NW}_k(\ESym_d) = \min\inparen{\binom{n}{d-k}, \binom{n}{k}}
\]
Hence, for $k = d/2$, we get 
\[
\CM{NW}_k(\ESym_d) = \binom{n}{d/2}.
\]
\end{claim}

\noindent 
The theorem follows directly from these two claims. \qedhere
\end{proof}

\section{Handling \emph{few} high degree gates}

In the last section we saw a way to prove lower bounds for $\SPS$ circuits were the degree of each product of linear forms was bounded by $d$. Suppose we had a $\SPS$ circuit $C$ such that all but say two of the products of linear forms have degree bounded by $d$. That is,
\[
C = C' + T_1 + T_2
\]
where $C' \in \Sigma\Pi^{[d]}\Sigma$ and  $T_1,T_2$ is a product of linear forms of degree possibly larger than $d$. Can we prove lower bounds for such circuits as well? \\

{\bf Key Idea: } Replace some variables by linear functions in other variables to make $T_1$ and $T_2$ equal to zero. \\

Say $T_1$ had one of the linear polynomials as $x_1 + 3x_2 + 5x_3 - 4$, then we shall replace $x_1$ by $- (3x_2 + 5x_3 - 4)$.
The result is that $T_1$ now becomes zero.
What happens to $T_2$?
Note that $T_2$ after this substitution still remains a product of linear polynomials, and its degree cannot increase in this process.
However, it may be the case that $T_2$ was $(x_1 + 3x_2 + 5x_3 - 7)^{2d}$ and under our substitution of $x_1$ this reduces to a constant.
But this is still good because the goal is to eliminate all high degree $T_i$s completely (either by making them zero, or reducing them to constants).
This was the key idea of Shpilka and Wigderson~\cite{sw2001} and we shall formalize this as a lemma.

\begin{lemma}\label{lem:d3-few-affine-projection}
  Let $C = T_1 + \cdots + T_s$ be a $\Sigma\Pi\Sigma$ and suppose $r$ of the $T_i$s have degree greater than $d$.
Then by taking an \emph{affine projection of co-dimension at most $r$}, that is by setting at most $r$ variables to linear functions in the remaining variables, the resulting circuit $C' = T_1' + \cdots T_{s'}'$ is a $\Sigma\Pi^{[d]}\Sigma$ circuit with $s' \leq s$.
\end{lemma}

In order to prove a lower bound for $\Sigma\Pi^{[d]}\Sigma$ circuits, we needed to find a polynomial $f$ for which $\dim \partial^{=k}(f)$ is large.
The strategy to prove lower bounds for $\Sigma\Pi\Sigma$ circuits with $r$ or fewer \emph{high degree} $T_i$s is now clear:
\begin{quote}
  Find a polynomial $g$ such that for every $g'$ that is an \emph{affine projection of co-dimension $r$} on $g$ (that is, obtained from $g$ by setting at most $r$ variables to linear functions in the remaining), we have that $\dim \partial^{=k}(g')$ is large.  
\end{quote}

\section{Shpilka and Wigderson's lower bound for $\Sigma\Pi\Sigma$ circuits}

Shpilka and Wigderson~\cite{sw2001} show that $\ESym_d$ not only has a large $\dim \partial^{=k}$, it also has large $\dim \partial^{=k}$ even after affine projections of co-dimension $d/100$. 

\begin{theorem}[\cite{sw2001}]\label{thm:SW-Esym-robust} Consider the polynomial $\ESym_d$ for any $d < \frac{n}{100}$. If $g$ is an affine projection of $\ESym_d$ of co-dimension $r < \frac{d}{100}$, then
\[
\dim \partial^{=k}(g) \spaced{\geq} \min\inparen{\binom{n-2r}{k},\binom{n-2r}{d-2r -k}}.
\]
Thus, if $k = (d-2r)/2$, then 
\[
\dim \partial^{=k}(g) \geq \binom{n-2r}{(d-2r)/2}.
\]
\end{theorem}

We shall defer the proof of \autoref{thm:SW-Esym-robust} to the end of this section but see why the above theorem implies \autoref{thm:SW-SPS-main-thm}. 

\begin{proof}[Proof of \autoref{thm:SW-SPS-main-thm}]
Consider the polynomial $\ESym_d$ for $d = \frac{n}{100}$. Suppose it is computable by a $\Sigma\Pi\Sigma$ circuit $C = T_1 + \cdots + T_s$ with at most $\frac{d^2}{100}$ wires. Let $r$ be the number of $T_i$s of degree more than $d$. Note that, given the bound on the number of wires, there cannot be more than $\frac{d}{100}$ $T_i$s that have degree more than $d$. Hence, we know that $r \leq \frac{d}{100}$. 

Now that $r \leq \frac{d}{100}$, by \autoref{lem:d3-few-affine-projection}, there is an affine projection $\rho$ of co-dimension at most $r$ such that 
\[
\rho(\ESym_d) \spaced{=} T_1' + \cdots + T_s' \spaced{\in} \Sigma\Pi^{[d]}\Sigma
\]
with $\deg(T_i') \leq d$ and $s \leq \frac{d^2}{100}$. 

But then, on the one hand we know from \autoref{claim:d3-nw-upperbound} that $\CM{NW}_k(\rho(\ESym_d))$ is at most $s \cdot \binom{d}{k}$ but on the other hand \autoref{thm:SW-Esym-robust} states that $\CM{NW}_k(\rho(\ESym_d))$ is \emph{large}. If we set the value of $k$ right ($k=(d-2r)/2$ should work) we get a contradiction to the original assumption that $s < \frac{d^2}{100}$. 

Hence, $s > \frac{d^2}{100} = \Omega(n^2)$ as claimed by the theorem. 
\end{proof}

This entire discussion can be summarized in the following very general remark.

\begin{mdframed}
\begin{remark}\label{rem:meta-SW-lb}
  Suppose we have a measure $\Gamma$ to prove lower bounds for a degree $d$ polynomial computed by a $\Sigma\Pi^{[D]}\Sigma$ circuit. If we can find an explicit polynomial $f$ of degree $d$ with the following properties,
  \begin{itemize}
  \item $\Gamma(f)$ is \emph{large},
  \item even after an affine restriction $\rho$ of co-dimension $r$, we have $\Gamma(\rho(f))$ is \emph{large}
  \end{itemize}
  Then, we can get a $\Omega(Dr)$ lower bound for $f$. 
\end{remark}
\end{mdframed}
What we did above was choose the polynomial $f$ as $\ESym_d$ with $D = d = \Omega(n)$ and using the fact that the partial derivative space remains large even after an affine restriction of co-dimension $r = \Omega(d)$, we get a $\Omega(d^2) = \Omega(n^2)$ lower bound for $\SPS$ circuits.

However, this remark is quite general and in principle, if we could prove a lower bound for say $\Sigma\Pi^{[d^2]}\Sigma$ circuits using some measure, then we could potentially extend this to give a cubic lower bound using the affine projection idea.
In fact, Kayal, Saha and Tavenas \cite{kst16} (which was subsequently strengthened by Balaji, Limaye and Srinivasan \cite{BLS16}) do indeed show an \emph{almost} cubic lower bound using a measure that we shall be discussed later.
We shall see this later in the survey.

\begin{theorem}[\cite{kst16}] There is an explicit $n$-variate poylnomial $f$ of degree $d$ such that any $\SPS$ circuit computing $f$ must require $\Omega\inparen{n^3}{\poly \log(n)}$ size. 
\end{theorem}

Much later in the survey\todo{add link}, we shall see the proof of Balaji, Limaye and Srinivasan \cite{BLS16} that, besides being a much simpler and modular proof of \cite{kst16}, also proves the lower bound for a polynomial computed by a depth-$5$ circuit.

\subsection{Rough proof of Theorem~\ref{thm:SW-Esym-robust}}

Let us order the variables as $x_1 \succ x_2 \succ \cdots \succ x_n$.
Say we have an affine projection $\rho$ of co-dimension $r$ that sets the linear polynomials $\set{\ell_1,\ldots, \ell_r}$ to zero. Let $x_{i_j}$ be the \emph{highest} variables participating in each $\ell_j$, that is,
\[
  \ell_j \spaced{=} x_{i_j} - \ell_j'.
\]
Thus, applying this affine restriction is equivalent to replacing each $x_{i_j}$ by $\ell_{j}'$. To get a sense of what this does to $\ESym_d$, let $y_j$ be the \emph{highest variable} participating in $\ell_j'$.

Let us now look at $\ESym_d$, paying attention to the variables $x_{i_j}$s and $y_j$s. We may assume that  $S = \set{x_{i_1},\ldots, x_{i_r}, y_1,\ldots, y_r}$ are all distinct variables (why?).
\begin{eqnarray*}
  \ESym_d(x_1,\ldots, x_n) & = &  x_{i_1}\cdots x_{i_r}\cdot y_1\cdots y_r \cdot \ESym_{d - 2r}(\vecx \setminus S) \spaced{+} \text{other monomials}
\end{eqnarray*}
Replacing each $x_{i_j}$ by $\ell_j'$ introduces higher powers of $y_1,\ldots, y_r$.

\begin{claim}
  If we collect all monomials in $\rho(\ESym_d)$ that are divisible by $y_1^2\cdots y_r^2$, it is precisely
  \[
    y_1^2 \cdots y_r^2 \cdot \ESym_{d-2r}(\vecx \setminus S).
  \]
\end{claim}

\begin{exercise}
Prove this claim. 
\end{exercise}

That is, such monomials can \emph{only} be generated by the first term in the above equation. Now, if we only choose to differentiate by monomials in $\vecx \setminus S$, the remaining monomials do not interfere with the monomials that are divisible by $y_1^2 \cdots y_r^2$. Therefore we get
\[
  \dim \partial^{=k}(\rho(\ESym_d)) \;\geq\; \dim \partial^{=k}\ESym_{d-2r}(\vecx \setminus S) \;=\; \min\inparen{\binom{n-2r}{d-2r-k}, \binom{n-2r}{k}}.\qed
\]

% \section{Revisiting the hom. $\SPS$ lower bound of \cite{nw1997}}

% From \autoref{obs:low-rank-sps-rank}, if $T = \ell_1 \cdots \ell_d$ then for every $0\leq k \leq d$ we have
% \[
% \dim \partial^{=k}(T) \spaced{\leq} \binom{d}{k}. 
% \]
% For a polynomial such as the $\Det_n$, we know that $\partial^{=k}(\Det_n) = \binom{n}{k}^2$. Let us first get a sense of what is the best $k$ to choose for proving lower bounds for $\Sigma\Pi^{[d]}\Sigma$ circuits. We recall some bounds on binomial coefficients from \autoref{chap:binom-estimates}. 

% \begin{proposition*}[\autoref{prop:binom-ub-lb}]For any $n\geq k \geq 0$, 
% \[
% \pfrac{n}{k}^k \spaced{\leq} \binom{n}{k} \spaced{\leq} \pfrac{ne}{k}^k\qedhere
% \]
% \end{proposition*}

% \begin{lemma}[\cite{nw1997}]\label{NW:d3hom-mainlemma} For any $d \geq 0$, any $\Sigma\Pi^{[d]}\Sigma$ circuit computing $\Det_n$ must have size $\exp\inparen{\Omega\pfrac{n^2}{d}}$. 
% \end{lemma}
% \begin{proof}
% As just mentioned, if $C = T_1 + \cdots + T_s$ where each $T_i$ is a product of at most $d$ linear polynomials, then 
% \[
% \dim(\partial^{=k}(C)) \spaced{\leq } s \cdot \binom{d}{k}.
% \]
% On the other hand, 
% \[
% \dim(\partial^{=k}(\Det_n)) \spaced{\geq } \binom{n}{k}^2.
% \]
% Therefore, we get a lower bound of 
% \begin{eqnarray*}
% s & \geq & \frac{\binom{n}{k}^2}{\binom{d}{k}}\\
%   & \geq & \pfrac{n}{k}^{2k} / \pfrac{de}{k}^k \spaced{=} \pfrac{n^2}{e\cdot dk}^k
% \end{eqnarray*}
% Thus, by choosing $k = \pfrac{n^2}{2ed}$, we get a lower bound of $2^k = \exp\inparen{\Omega\pfrac{n^2}{d}}$.
% \end{proof}




% \section{The win-win proof of \cite{sw2001}}

% \begin{proof}[Proof of \autoref{thm:SW-SPS-main-thm}]
%  Say $\Det_n$ is computed by $C = T_1 + \cdots + T_s$ where each $T_i$ is a product of linear forms.
% We will prove by induction that $s\geq \frac{n^4}{100\log n} + 1$. Starting with $C$, we shall modify the circuit to a new circuit $C'$ that computes $\Det_{n-1}$ (up to renaming variables). By induction, we shall assume that 
% \[
% \mathrm{size}(C') \spaced{\geq} \frac{(n-1)^4}{100\log (n-1)}.
% \]

% We shall keep in mind a threshold $d$ that would be set shortly (spoiler: $d= n^2/20\log n)$).
% By \autoref{NW:d3hom-mainlemma} we know that if each $T_i$ is a product of at most $d$ linear polynomials, we have a lower bound of $\exp(\Omega(n^2/d))$. We shall say a variable $x_{ij}$ is \emph{$d$-good} in $C$ if each $T_i$ that depends on $x_{ij}$ has degree at most $d$. \\

% {\bf Case 1:} There is a variable in the first row $x_{1j}$ that is $d$-good. 

% \medskip

% \noindent 
% In this setting, consider the polynomial $C' = C_{(x_{1i} = 1)} - C_{(x_{1i} = 0)}$.
% As far as $\Det_n$ is concerned, this is equivalent to taking the derivative with respect to $x_{ij}$ and hence the polynomial is the corresponding $(n-1)\times (n-1)$ minor.
% The circuit $C'$ on the other hand now is a $\Sigma\Pi^{[d]}\Sigma$ circuit with top fan-in at most $2s$, as all $T_i$s in $C$ that do not depend on $x_{ij}$ would now be eliminated.
% By \autoref{NW:d3hom-mainlemma}, this implies that $2s > \exp(\Omega((n-1)^2/d))$ which is certainly bigger than $(n^4 / \log n)$ if $d = n^2/20\log n$.\\

% {\bf Case 2:} There is \emph{no} variable in the first row that is $d$-good. 

% \medskip

% \noindent 
% We know that the variable $x_{12}$ is not $d$-good.
% This means that there is some $T_i$ of degree more than $d$ which has a factor of the form $(x_{12} - \ell)$.
% We shall now set $x_{12} = \ell$ in $C$ and this eliminates the gate $T_i$ altogether and thereby reducing the size by at least $d$ wires. 

% Now move on to $x_{13}$. It is possible now that after eliminating $T_i$ in the previous step, we may now have $x_{13}$ to be $d$-good. Then we reduce to Case 1 and we are done. Otherwise, there is some $T_j$ of degree more than $d$ with a factor of the form $(x_{13} - \ell')$. Once again, we set $x_{13} = \ell'$ to eliminate $T_j$ and thereby reduce the size of the circuit by $d$. \\

% Repeating this process for $\set{x_{12}, \ldots, x_{1n}}$, we either reach Case 1 (in which case we are done), or we have thereby reduced the circuit by at least $d(n-1)$ wires. So far, we have only messed with the variables $\set{x_{12},\ldots, x_{1n}}$ but this is not an issue because we can further set $x_{11} = 1$ and $x_{21} = x_{31} = \cdots = x_{n1} = 0$. Therefore, the polynomial thus computed has to be the minor with respect to $x_{11}$ which is just $\Det_{n-1}$ up to renaming variables. But then,
% \[
% \mathrm{size}(C) \spaced{\geq} \mathrm{size}(C') \;+\; \pfrac{(n-1)^2 n^2}{20\log n}
% \]
% Hence, it follows by induction that $\mathrm{size}(C) \geq \pfrac{n^4}{100\log n} + 1$.
% \end{proof}


%%% Local Variables: 
%%% mode: latex
%%% TeX-master: "fancymain"
%%% End: 
