\chapter{Limitations of sub-additive rank methods}\label{chap:subadditive-rank-barrier}

So far, almost all lower bound proofs that we have seen follow this template:

\begin{enumerate}
\item Identify a set $\mathcal{B}$ of \emph{building blocks} or \emph{simple polynomials}.
\item Build a function $\Gamma: \F[\vecx] \rightarrow \N$, where $\Gamma(f)$ is the rank of an associated matrix $M(f)$ whose entries are linear functions in the coefficients of $f$.
\item Show that for each building block $g \in \mathcal{B}$ we have that $\rank(M(g)) \leq U$.
\item Find an explicit polynomial $f$ such that $\rank(M(f)) \geq L$.
\item This shows that if $f = g_1 + \cdots + g_s$ where each $g_i \in \mathcal{B}$, then $s \geq L/U$. 
\end{enumerate}

\noindent
For such a template, what is the largest we can make the ratio $L/U$? This would give us an indication of the limitations of techniques that use subadditivity via building blocks to prove the required lower bound. A beautiful result of Efremenko, Garg, Oliveira and Wigderson~\cite{EGOW18} prove \emph{unconditional} limitations of these techniques for the instances of Waring rank\footnote{Waring rank of a polynomial is the top fan-in of the smallest $\Sigma\!\wedge\!\Sigma$-circuit computing it.} and tensor rank and we shall see their proof in this chapter. Throughout this chapter, the field $\F$ will be assumed to have characteristic zero.
\begin{theorem}[Limitations for Waring Rank]\label{thm:limitations-waring-rank}
  Let $M: \F[\vecx]_{= d} \rightarrow \operatorname{Matrix}(\F)$ be a map that assigns a matrix to every homogeneous polynomial of degree  $d$,  that acts linearly, that is $M(f + g) = M(f) + M(g)$. Suppose $\mathcal{B} = \setdef{\ell^d}{\ell = \sum a_i x_i}$; define $\rank(\mathcal{B}) = \max\setdef{\rank(g)}{g\in \mathcal{B}}$. Then, for any $f \in \F[\vecx]_{=d}$ we have that
  \[
    \frac{\rank(M(f))}{\rank(M(\mathcal{B}))} \leq (d+1) \cdot \binom{n+(d/2)}{n}. 
  \]
\end{theorem}

This shows that such sub-additive rank measures as outlined above cannot prove a lower bound for $\Sigma\!\wedge\!\Sigma$-circuits much better than $\binom{n+(d/2)}{n}$, though counting arguments show that there are $n$-variate degree-$d$ polynomials that require $\Sigma\!\wedge\!\Sigma$-circuits of size $\frac{1}{\poly(n)} \cdot \binom{n+d}{n}$ to compute it.  

Efremenko, Garg, Oliveira and Wigderson~\cite{EGOW18} also show a similar limitation for tensor rank. 
\begin{theorem}[Limitations for Tensor Rank]\label{thm:limitations-tensor-rank}
  Let $X = X_1 \sqcup \cdots \sqcup X_d$ with each $|X_i| = n$, and let $\F[X]_{\text{SML}}$ refer to the class of set-multilinear polynomials with respect to the above partition (these are synonymous to tensors of order-$d$).
  
  Let $M: \F[\vecx]_{\text{SML}} \rightarrow \operatorname{Matrix}(\F)$ be a map that assigns a matrix to every set-multilinear polynomial  that acts linearly, that is $M(f + g) = M(f) + M(g)$. Suppose
  \[\mathcal{B} = \setdef{\ell_1(X_1)\cdots \ell_d(X_d)}{\ell_i = \sum a_{ij} x_{ij}},\]
  which is synonymous to rank-$1$ tensors. Define $\rank(\mathcal{B}) = \max\setdef{\rank(g)}{g\in \mathcal{B}}$. Then, for any $f \in \F[\vecx]_{\text{SML}}$ we have that
  \[
    \frac{\rank(M(f))}{\rank(M(\mathcal{B}))} \leq 2^{d} \cdot  n^{\floor{d/2}}. 
  \]
\end{theorem}

Once again, a counting argument would tell us that there are order-$d$ tensors of rank $\frac{1}{d} \cdot n^{d-1}$. Furthermore, both in the setting of $\Sigma\!\wedge\!\Sigma$-circuits and tensor rank, current vanilla techniques achieve a lower bound of $n^{\floor{d/2}}$. The above theorems show that this template of proving lower bounds cannot do too much better. 


We shall see the proof of \autoref{thm:limitations-waring-rank} in complete detail; the proof of \autoref{thm:limitations-tensor-rank} is very similar. 

\subsection*{What does this limitation mean?}

Currently, the limitations hold for tensor rank and Waring rank but it is conceivable that there is a broader setting where such limitations can be proven. What does that imply for algebraic circuit lower bounds?

Almost all the lower bounds we have discussed so far follows the template of constructing a suitable linear matrix, upper bounding the rank of some building blocks, and lower bounding the rank for the target polynomial. There are, nevertheless, a few exceptions that we have seen in this survey. The first is the lower bound of Grigoriev and Karpinski~\cite{grigoriev98} that we saw in \autoref{chap:GK}. The complexity measure used by Grigoriev and Karpinski~\cite{gr00} is indeed via a linear matrix, but we study the rank of this matrix \emph{after a certain set of columns have been dropped}. It is unclear if such non-deterministic choices can also be captured in the framework of Efremenko, Garg, Oliveira and Wigderson.

Another lower bound that does not go via sub-additive rank measures is the determinantal complexity lower bound of Mignon and Ressayre~\cite{mr04} (as seen in \autoref{chap:dc}). This lower bound, though is again a rank of an associated matrix, is not proven using sub-additivity of building blocks.

Furthermore, it should also be noted that even for Waring rank and tensor rank, we \emph{can} provide super-polynomial lower bounds via sub-additive rank measures. Of course, in the tensor-rank setting, a lower bound of $n^{\floor{d/2}}$ is trivial since this just involves flattening to a matrix. But for most other applications, we are aiming for much weaker lower bounds than $\binom{n+d}{d}$. Hence, even if this framework extends for models such as homogeneous $\SPSP$ circuits, we cannot rule out the possibility that sub-additive rank methods can give lower bounds of $n^{\omega(\sqrt{d})}$, which is sufficient to separate $\VP$ and $\VNP$. 

\section*{Proof outline}

We have a set $\mathcal{B}$ such that every polynomial we care about can be expressed as a linear combination of polynomials in $\mathcal{B}$. We want to think of $\mathcal{B}$ as a set of \emph{simple polynomials} in the sense that  $\rank(\mathcal{B}) = U$.

Now suppose that the set $\mathcal{B}$ of \emph{simple polynomials} can be generated by one polynomial $B(\vecy)$ with $\abs{\vecy} = m$, that is for every $g\in \mathcal{B}$ there is some $\veca \in \F^m$ such that $g = B(\veca)$. In the case of Waring rank, this polynomial would be \[B(\vecy) = (x_1 y_1 + \cdots + x_ny_n)^d,\] and in the case of tensor-rank this polynomial would be
\[ B(\vecy) = \inparen{\sum_{j=1}^n y_{1j} x_{1j}} \times \cdots \times \inparen{\sum_{j=1}^n y_{dj} x_{dj}}. 
  \]
  The fact that $\max_\veca \rank(M(B(\veca))) = U$ implies that the rank of the symbolic matrix $M(B(\vecy))$ \emph{over the function field $\F(\vecy)$} is $U$. Now suppose $f$ was an arbitrary polynomial, we know that $f(\vecx) = \sum_{\veca\in A} B(\veca)$ for some finite set $A$ since $\mathcal{B}$ forms a spanning set. Therefore, we have
  \[
    M(f) = \sum_{\veca\in A} M(B(\veca)).
  \]
  What we know is that any fixed evaluation $M(B(\veca))$ of $M(B(\veca))$ has rank at most $U$ and hence its rows/columns are spanned by some set of $U$ row/column vectors. Suppose it turns out that there are row/column vectors $V_1,\ldots, V_{t}$ (with $t$ not much larger than $U$) such that \emph{every evaluation} $M(B(\veca))$ has its rows/columns spanned by these $t$ vectors. Then, by linearity, so must the rows/columns of $M(f)$! This would let us show that $\rank(M(f)) \leq t$  for any arbitrary $f$. Efremenko, Garg, Oliveira and Wigderson show that something almost similar is true for the setting of Waring rank or tensor rank.

  \begin{lemma}[Bounding rank of all evaluations] \label{lem:limitation:decomposition-eval-space}
    Consider the set-up as before for the Waring rank or tensor rank setting. Suppose $\rank_{\F(\vecy)}(M(B(\vecy)) = U$. Then there exists a set of row vectors $R_1,\ldots, R_t$ and column vectors $C_1,\ldots, C_{t'}$ such that every evaluation $M(B(\veca)) = C + R$ where the columns of $C$ are spanned by $\set{C_1,\ldots, C_{t'}}$ and the rows of $R$ are spanned by $\set{R_1,\ldots, R_t}$.

    \begin{itemize}
    \item  For the setting of Waring rank, we have $t + t' \leq U \cdot (d+1) \cdot \binom{n+(d/2)}{n}$.

    \item For the setting of tensor rank, we have $t+t' \leq U \cdot 2^{d} \cdot n^{\floor{d/2}}$.
    \end{itemize}
  \end{lemma}

  This lemma immedietely yields that $\rank(M(f)) \leq U\cdot (d+1) \cdot \binom{n+(d/2)}{n}$ in the Waring rank setting, and $\rank(M(f)) \leq U\cdot 2^{d} \cdot n^{\floor{d/2}}$ in the tensor rank setting. \\

  In fact, as it would soon be evident from the proof, the same set-up can be used to prove similar limitations in a more general setting. Suppose $B(\vecy)$ is the polynomial whose evaluations generate the class $\mathcal{B}$ and suppose $\deg_\vecy B(\vecy) = \deg f$ and $\abs{\vecy} = m$. Then for any $f \in \operatorname{span}(\mathcal{B})$ we have
  \[
    \frac{\rank(M(f))}{\rank(M(\mathcal{B}))} \leq (d + 1) \cdot \binom{m + \floor{d/2}}{m}. 
  \]
  The Waring rank setting was the instantiation with $m = n$. Of course, if $m = \Omega(n^2)$, then we get a meaningless limitation that these techniques cannot prove lower bounds better than $n^d$ (which is roughly the number of monomials in $f$ anyway!) but we get a meaningful limitation for all $m = n^{2 - \epsilon}$. (In fact, the tensor rank setting can also be cast this way with $m = dn$, but \autoref{thm:limitations-tensor-rank} improves this above bound.)\\

  In the rest of this chapter, we shall see a proof of the key lemma (\autoref{lem:limitation:decomposition-eval-space}) for the general instance when the building blocks are generated by a $B(\vecy)$.

\section{The main decomposition lemma}

We have a polynomial $B(\vecx, \vecy) \in \F[\vecx, \vecy]$ that is homogeneous in $\vecy$ and $\deg_\vecy B = \deg f = d$; let $\abs{\vecy} = m$ and $\abs{\vecx} = n$. We shall sometime denote $B(\vecx, \vecy)$ by just $B(\vecy)$ as we will be more interested in $B$ as a function of $\vecy$.
\[
  \mathcal{B} = \setdef{B(\veca)}{\veca \in \F^m}
\]
Let us assume that $M(f)$ is an $N\times N$ matrix, without loss of generality, for some $N \geq 0$. 
If $U = \rank(M(\mathcal{B}))$, then we have that $\tilde{M} = \rank_{\F(\vecy)}(M(B(\vecy))) = U$ as a symbolic matrix. Therefore, $\tilde{M}$ can be written as
\[
  \tilde{M} = V_1 W_1^T + \cdots V_U W_U^T
\]
for $N$-length vectors $V_i, W_i$ with entries from $\F(\vecy)$. This is a little annoying that even though $\tilde{M}$ only involves entries that are homogeneous degree $d$ polynomials in $\vecy$, the entries of the vectors could involve rational functions in $\vecy$ of arbitrary numerators and denominators. A natural question is whether every \emph{homogeneous matrix} of small rank has a \emph{small homogeneous rank-$1$ decomposition}. Efremenko, Garg, Oliveira and Wigderson show that this is indeed true.

\begin{lemma}[Homogeneous rank-$1$ decomposition]\label{lem:limitation:hom-rank-decomposition}
  Suppose $\tilde{M}$ is a matrix with entries in $\F[\vecy]_{=d}$ and suppose $\rank_{\F(\vecy)} \tilde{M} = U$. Then there are row vectors $P_1,\ldots, P_{t}$ and $Q_1,\ldots, Q_t$ such that
  \[
    \tilde{M} = P_1 Q_1^T + \cdots + P_t Q_t^T,
  \]
  satisfying
  \begin{itemize}
  \item $t \leq (d+1) U$,
  \item For each $i$ there is some $0 \leq d_i \leq d$ such the vectors $P_i$ and $Q_i$ consists of homogeneous polynomials of degree $d_i$ and $d - d_i$ respectively.  
  \end{itemize}
\end{lemma}

In other words, if the rank of a matrix $\tilde{M}$ involving homogeneous polynomials of degree $d$ as its entries is at most $U$, then its \emph{homogeneous rank}\footnote{Defined analogously as the smallest \emph{homogeneous} rank-$1$ decomposition} is at most $(d+1) U$. 

An analogue of the above lemma holds for set-multilinear polynomials as well, where we would like to have each rank-$1$ term to also be set-multilinear. This is the analogue required in the proof of \autoref{thm:limitations-tensor-rank}. 

The proof of \autoref{lem:limitation:hom-rank-decomposition} is along the lines of Strassen's division elimination and we will defer this to later and see how to prove \autoref{lem:limitation:decomposition-eval-space} using this.

\begin{proof}[Proof of \autoref{lem:limitation:decomposition-eval-space}]
  Consider the homogeneous rank-$1$ decomposition of $\tilde{M}$ given above by \autoref{lem:limitation:hom-rank-decomposition}:
  \[
    \tilde{M} = P_1 Q_1^T + \cdots + P_t Q_t^T.
  \]
  Write $\tilde{M} = \tilde{R} + \tilde{C}$ where
  \[
\tilde{R} := \sum_{i: \deg P_i \leq \frac{d}{2}} P_i Q_i^T\quad,\quad \tilde{C} := \sum_{i: \deg P_i > \frac{d}{2}} P_i Q_i^T.
  \]
  We will focus on $\tilde{R}$; the analysis for $\tilde{C}$ would be analogous.
  \begin{align*}
    \tilde{R}(\vecy) & = \sum_{i: \deg P_i \leq \frac{d}{2}} P_i(\vecy)  Q_i(\vecy)^T\\
             & = \sum_{i: \deg P_i \leq \frac{d}{2}}\;\; \sum_{\vece: \abs{\vece} \leq \frac{d}{2}} \vecy^{\vece} \cdot \coeff_{\vecy^{\vece}}(P_i)  Q_i (\vecy)^T\\
    \implies \tilde{R}(\veca) & = \sum_{i: \deg P_i \leq \frac{d}{2}}\;\; \sum_{\vece: \abs{\vece} \leq \frac{d}{2}} \veca^{\vece} \cdot \coeff_{\vecy^{\vece}}(P_i)  Q_i (\veca)^T
  \end{align*}
  Hence, every row of $\tilde{R}(\veca)$ is spanned by $\setdef{\coeff_{\vecy^{\vece}}(P_i)}{\deg P_i \leq \frac{d}{2}\;,\; \deg \vecy^{\vece} \leq \frac{d}{2}}.$  Similarly, every column of $\tilde{C}(\veca)$ is spanned by $\setdef{\coeff_{\vecy^{\vece}}(Q_i)^T}{\deg P_i > \frac{d}{2}\;,\; \deg \vecy^{\vece} \leq \frac{d}{2}}.$ Together, we have at most $t \cdot \binom{m + (d/2)}{m}$ vectors. 
The statement of \autoref{lem:limitation:decomposition-eval-space} follows since $M(f)$ is a linear combination of the evaluations $\setdef{R(\veca)+C(\veca)}{\veca\in \F^m}$   
\end{proof}

\subsection*{Proof of the decomposition lemma}

We begin with the standard rank-$1$ decomposition of the symbolic matrix $\tilde{M}$.
\[
  \tilde{M} = V_1 W_1^T + \cdots V_U W_U^T
\]
As stated, the entries of $V_i,W_i$ could involve rational functions in $\vecy$. Nevertheless, we can clear denominators and obtain
\[
\tilde{M} = \pfrac{1}{g(\vecy)} \cdot \inparen{V'_1 W'_1{}^T + \cdots V'_U W'_U{}^T}
\]
where now $g(\vecy)$ and the entries of $V_i'$ and $W_i'$ are honest-to-god polynomials in $\vecy$. Assume without loss of generality that $g(\mathbf{0}) = 1$ (by suitable translation and scaling if necessary). Then, $g(\vecy) = 1 - g'(\vecy)$ where $g'(\mathbf{0}) = 0$. Therefore,
\[
\frac{1}{g(\vecy)} = 1 + g'(\vecy) + (g'(\vecy))^2 + \cdots . 
\]
Therefore, 
\[
  \tilde{M} = \inparen{V'_1 W'_1{}^T + \cdots V'_U W'_U{}^T} \cdot \inparen{1 + g'(\vecy) + (g'(\vecy))^2 + \cdots}.
\]
Here comes the important point: even though the right hand side has infinitely many terms, every entry on the left-hand side has degree at most $d$. Hence, we may just collect homogeneous components from the RHS to obtain our homogeneous decomposition. Define $\tilde{g}(\vecy) = 1 + g'(\vecy) + \cdots + (g'(\vecy))^d$. Then,
\[
  \tilde{M} = \sum_{i=0}^d \sum_{j=1}^U \inparen{\operatorname{Hom}_{ i}\inparen{\tilde{g} \cdot V_j'}} \inparen{\operatorname{Hom}_{d - i}\inparen{W_j'}}^T. \qed
\]


%%% Local Variables:
%%% mode: latex
%%% TeX-master: "fancymain"
%%% End:
